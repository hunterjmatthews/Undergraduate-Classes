\documentclass{report}
\usepackage{tikz}
\usetikzlibrary{shapes.geometric}
\usetikzlibrary{decorations, decorations.markings} 
\tikzset{corner/.style={
  postaction={decorate},
  decoration={markings,mark=between positions 0 and 1 step 0.2 with {\fill (0,0) circle (2pt);}}
  }
}

\input{preamble}
\input{macros}
\input{letterfonts}

\title{\Huge{Math 4301}\\Quiz 1}
\author{\huge{Hunter Matthews}}
\date{05/25/23}

\begin{document}

\maketitle
\newpage% or \cleardoublepage
% \pdfbookmark[<level>]{<title>}{<dest>}
% -- Question #1 -- 
\qs{}{Let $A_{n}=\{x\in \bbQ: -3+\dfrac{1}{4n^2+n-1}<x<\pi+\dfrac{(-1)^n}{3n-1}\}, n\in \bbN$ and $A=\bigcup_{n=1}^{\infty} A_n$. Find inf(A)}
\pf{Solution}{The answer is $-3$.}
% -- Question #2 --
\qs{}{Let $A_{n}=\{x\in \bbR: 1-\dfrac{1}{n^2}<x<6+\dfrac{\sqrt{2}}{n^2+n+5}\}$ and $A=\bigcap_{n=1}^{\infty} A_n$. Find sup(A)}
\pf{Solution}{The answer is $6$}
% -- Question #3 --
\qs{}{Choose all correct answers (there might be more than one correct answer)\\
(a) An ordered Field $\bbF$ has archimedean property if and only if for all $x\in \bbF, x > 0$ there is $n\in \bbN$ such that $0<\dfrac{1}{n}<x$\\
\\(b) The set $\bbR$ of all real numbers is a complete ordered field.\\
\\(c) The set $s=\{x\in \bbQ: 0<x<\sqrt{2}\}\subseteq \bbR$ has both the greatest and the lest upper bounds in $\bbR$.\\
\\(d) The set of all real numbers $\bbR$ does not satisfy archimedean property.\\
\\(e) The set $S=\{\dfrac{n^2+(-1)^n}{n}: n\in \bbN\}\subseteq \bbR$ is bounded above.}
\pf{Solution}{The answers are A,B, and C.}
% -- Question #4 --
\qs{}{Choose all correct answers (there might be more than one correct answer)\\
(a) Well-ordering property and the Principle of Mathematical induction are equivalent statements for the set of all natural numbers.\\
\\(b) The set $\bbZ$ of all integers has the well-ordering property.\\
\\(c) The set $A - \{x\in \bbQ: 0 < x < 1\}$ has the greatest and the least upper bounds in $\bbQ$.\\
\\(d) The set of $\bbQ$ of all rational numbers is a complete ordered field.\\
\\(e) If $s\subseteq \bbZ, 0\in S$ and $(k-1),(k+1)\in S$ whenever $k\in S$, then $S=\bbZ$.}
\pf{Solution}{The answers are A,C, and E.}
\end{document}
